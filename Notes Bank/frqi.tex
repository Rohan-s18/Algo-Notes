\documentclass[letterpaper, 12pt]{article}

\usepackage{amssymb}


\title{Using Flexible Representation of Quantum Images for Classical Image Compression}
\author{Rohan Singh}
\date{April 24, 2023}

\begin{document}

\maketitle

\section{Abstract}
Flexible Representation of Quantum Images (FRQI) is a technique for representing and processing classical images in quantum computing. FRQI involves the principles of quantum mechanics such as superposition and entanglement to help speed up classical image processing algorithms.\\\\
More specifically, FRQI represents classical images as superpositions of quantum states. This is done by representing every pixel in the image as a quantum state, with its corresponding coefficient $C_i$ of the state corresponding to the pixel's intensity. The images are then represented as a linear combination of these states.\\\\
In this paper we will be looking at the compression of classical images using FRQI techniques of image representation.

\section{FRQI Image representation}
As you may already know, in classic computing we represent images as a matrix of pixels, where a pixel at location $(x,y)$ has a designated value for intensity (rgb/greyscale etc.). In a nutshell, any sort of image processing done in classical computing is simply a manipulation of the matrix representing the pixels and the pixel values themselves. So before we get into the image compression, it is important to understand what a quantum image is and how can we represent it.

\subsection{Images as Quantum States}
We can represent a classical image using FRQI as:
$$
|\psi\rangle = \frac{1}{\sqrt{N^2}} \sum_{x,y} I(x,y) |x,y\rangle
$$
In the eqn above, $|\,x,y\rangle$ is the basis state corresponding to the classical pixel at location $(x,y)$ in the image that we have, $I(x,y)$ is the intensity of the pixel $(x,y)$, additionally $N$ is the size, the following linear combination of the basis states yields us $|\psi\rangle$ which represents the quantum state of the classical image.

\subsection{Image reconstruction from Quantum States}
Similarly, in order to reconstruct the classical image from $|\psi\rangle$ we must measure the quantum state in the computational basis and reconstruct it using the probabilities that we get for measuring it in all of the basis states.\\\\
Using Born's rule:
$$
P(x,y) = |\langle x,y|\psi\rangle|^2
$$\\
After measuring the quantum state $|\psi\rangle$ in the computational basis, the classical image can be reconstructed from the measured probabilities.\\

\section{Compression of Classical Images}
To compress an image we must first convert it into a quantum state. After doing that we use a \textit{Unitary Matrix} to make the state smaller while still keeping most of the important parts. Just as mentioned in the previous section, we then measure the image to get a new set of numbers to reconstruct the compressed image. 

\subsection{Steps for Compression of Images}
The steps to compress a classical image usign FRQI are:\\\\
1. We will first represent the classical image as a quantum state $|\psi\rangle$ using the FRQI representation:

$$
|\psi\rangle = \frac{1}{\sqrt{N^2}} \sum_{x,y} I(x,y) |x,y\rangle
$$
2. We will now apply a unitary transformation matrix $U$ that will compress the quantum state representing the image:
$$
|\phi\rangle = U|\psi\rangle
$$
3. We will then measure the compressed quantum state $|\phi\rangle$ in the computational basis of the image to obtain a reduced set of coefficients (here $k$ is the desired compression rate):

$$
\{p_1, p_2, ..., p_k\}
$$

\section{Decompression of Classical Images}
To decompress the compressed image we can simply use the same trick that we used to compress the image, however instead of using $U$ as the Unitary Matrix, we will be using $U^{-1}$ as the unitary matrix. 

\subsection{Steps for Decompression of Images}
We can decompress the compressed image using FRQI by:\\\\
1. We will use the set of coefficients for the compressed image $\{p_1, p_2, ..., p_k\}$ to reconstruct the compressed quantum state (here $|\phi_i\rangle$ is the $i$th eigenvector of the compression unitary matrix $U$):

$$
|\phi'\rangle = \sum_{i=1}^k \sqrt{p_i} |\phi_i\rangle
$$
2. We will then apply the inverse of the compression unitary matrix $U^{-1}$ (it exists because $U$ is a unitary matrix) to the compressed quantum state to obtain the original quantum state $|\psi'\rangle$:

$$
|\psi'\rangle = U^{-1}|\phi'\rangle
$$
3. We can obtain the classical image by measuring the quantum state $|\psi'\rangle$ in the computational basis. \\


\section{Mathematical Overview}
In this section we will be going over some of the mathematical tools used in this paper.

\subsection{Unitary Matrices}
A Unitary Matrix $U$ is defined as:\\
$U_m_\times_m$ $\in$ $\mathbb{C}^m^\times^m$ such that\\
$U$$U^\dag$ = $U^\dag$$U$ = $I$\\
One important thing about Unitary Matrices is the preservation of vector length.


\end{document}
